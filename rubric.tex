\documentclass{article}
\usepackage[czech]{babel}          % Language
\usepackage[utf8]{inputenc}        % Encoding of characters in this .tex file
\usepackage{cmap}                  % Make PDF file searchable and copyable (ASCII characters)
\usepackage{lmodern}               % Make PDF file searchable and copyable (Accented characters)
\usepackage[T1]{fontenc}           % Hyphenate accented words
\usepackage[a4paper, top=1cm, bottom=1cm, left=1cm, right=1cm]{geometry}  % Paper size and margins
\usepackage{multicol}              % Enable multiple columns
\usepackage[fontsize=10]{scrextend} % Font size
\usepackage{marginnote}            % Enable margin notes
\usepackage{rotating}              % Rotated environments (for margin headers)
\usepackage[protrusion]{microtype} % Better typeset results
\usepackage{enumitem}              % Nicer enumeration lists
% Hack follows: my version of csquotes does not know Czech (but Danish style is the same)
\usepackage[style=danish]{csquotes} % Add command to enquote strings
\usepackage{ragged2e}              % Add \justify command
\usepackage{setspace}              % Change line spacing
\usepackage{titlesec}              % Customize chapters
\usepackage{xspace}                % Automatic space after macros
\usepackage[usenames,dvipsnames,svgnames,table]{xcolor} % Custom colors
\usepackage{chronology}            % Timeline
\usepackage{tikz}                  % Number lines
\usepackage{url}                   % Enable URL typesetting

% ======= CUSTOM DOCUMENT SETTINGS ========

% Specify relevant Tikz libraries
\usetikzlibrary{arrows}

% \punct command allows to shift the footnotemark above a punctuation
\newlength{\spc} % declare a variable to save spacing value
\newcommand{\punct}[1]{%
  \settowidth{\spc}{#1}% set value of \spc variable to the width of #1 argument
  \addtolength{\spc}{-1.8\spc}% subtract from \spc about two (1.8) of its values making its magnitude negative
  #1% print the optional argument
  \hspace*{\spc}% print an additional negative spacing stored in \spc after #1
}

% Change \chapter and \section command display and spacings
% Source: https://tex.stackexchange.com/questions/63390/how-to-decrease-spacing-before-chapter-title
\titleformat{\section}[display]{\LARGE\bfseries}{\sectionname}{}{}
\titlespacing*{\section}{0em}{0.5em}{0em}
\titleformat{\subsection}[display]{\normalsize\bfseries}{\sectionname}{}{}
\titlespacing*{\subsection}{0em}{0.5em}{-0.5em}

% Remove headers and position page number in the bottom mid
\pagestyle{empty}

% Smaller spaces between enumeration lists
\setitemize{itemsep=0em}

% Disable automatic indentation, enable paragraph skip
\setlength{\parindent}{0em}
\setlength{\parskip}{0.5em}

% Rule length including the margin
\newlength{\rulelength}
\setlength{\rulelength}{\dimexpr(\textwidth+0.5cm)\relax}

% Smaller note (italics, gray)
\newcommand{\note}[1]{\textcolor{darkgray}{\small\itshape #1}}
\newcommand{\noteblack}[1]{{\small\itshape #1}}

% accurate absolute positioning
\newcommand{\putat}[3]{\begin{picture}(0,0)(0,0)\put(#1,#2){#3}\end{picture}}

\newcounter{rubricquestion}

% A single rubric criterion
\newcommand{\rubriccriterion}[4]{
\stepcounter{rubricquestion}
\subsection*{\therubricquestion: #1}

\smallskip
\note{Neznalý:} #2

\note{Začátečník:} #3

\note{Guru:} #4

\medskip
\begin{tikzpicture}
\draw (0,0) -- (8,0);
\foreach \i in {0,1,...,8} % numbers on line
{
\fill[black] (\i,0) circle (1.5 mm);
\fill[white] (\i,0) circle (1.4 mm);
}
\node at (0.15, -0.5) {neznalý};
\node at (3, -0.5)    {začátečník};
\node at (8, -0.5)    {guru};
\end{tikzpicture}
}

\newcommand{\teacherName}{Martin\xspace}

% ======= METADATA =======

\title{Učitelská rubrika}
\author{Kolektiv vyučujících Teaching labu}
\date{\today}

% ======= DOCUMENT START =======

\begin{document}

\section*{\teacherName: Učitelské znalosti a dovednosti}
\label{rubric}

\begin{onehalfspace}
\large 
Následující strany obsahují tzv. učitelskou rubriku. Touto rubrikou by tvůj cvičící (\teacherName) chtěl zjistit, jak vnímáš jeho/její schopnosti. Pro každou níže uvedenou učitelskou schopnost si přečti popisky \textit{Neznalý}, \textit{Začátečník} a \textit{Guru}. Pak se zamysli, jak na tom tvůj cvičící podle tebe je a zaznač to do nabídnuté škály. Celý dotazník je anonymní a zabere asi 10 minut. Odpovědi použije \teacherName ve spolupráci s vyučujícími kurzu pro cvičící \textit{Teaching Lab} pro zlepšení své výuky a výzkum kolem lepší výuky informatiky obecně.
\end{onehalfspace}
\bigskip

\begin{multicols*}{2}

\rubriccriterion{Cíle hodiny, vědomá pozornost}
{\teacherName působí, že nemá ve výuce jasně stanovené cíle. Když učí, vypadá, že nevnímá, co se děje a kam směřujeme. Často působí ztraceně.}
{\teacherName vypadá, že si někdy uvědomuje, jaký má cíl, co právě dělá a jaký to bude mít efekt. Vypadá ale, že většinou při výuce nemá vědomou pozornost.}
{Během výuky \teacherName působi, že si téměř vždy uvědomuje, jaký cíl sleduje, co se skupinou děje, co dělá a jaký to bude mít efekt. Ví, jak jsme se dostali tam, kde právě jsme.}

\vfill
\rule{\columnwidth}{0.4pt}
\vfill

\rubriccriterion{Interakce se skupinou, kladení otázek skupině}
{\teacherName se skupinou neinteraguje. Neptá se a vypadá, že neví, jak studenty zapojit.}
{\teacherName ví, že lze interagovat se skupinou a vypadá, že zná nástroje (chápe, jak by to šlo). Nedokáže je ale dobře používat. Někdy se skupiny ptá, ale nedostává odpověď.}
{\teacherName se skupinou často a efektivně interaguje. Ptá se způsobem, který studenty aktivizuje a zapojuje. Situace, kdy nedostává od skupiny odpověď dokáže obratně řešit (např. přeformulováním otázky).}

\vfill
\rule{\columnwidth}{0.4pt}
\vfill

\rubriccriterion{Strukturování výuky}
{\teacherName vypadá, že o strukturování výuky vůbec nepřemýšlí.}
{\teacherName působí, že chápe smysl přehledného strukturování své hodiny a snaží se o to. Často se ale zamotá, ztratí nit, nebo řeší více věcí naráz a studenti se pak ztrácí nebo odpojují.}
{Hodiny, které vede \teacherName mají jasnou strukturu. Studenti ví, co se právě děje, co bude následovat a chápou návaznosti. Mezi jednotlivými bloky vědomě dělá zřetelné přechody.}

\columnbreak

\rubriccriterion{Vlastní pocity a spokojenost s výukou}
{\teacherName působí nespokojeně se svými hodinami. Vypadá, že se netěší do výuky.}
{\teacherName vypadá, že si ve výuce často moc nevěří. Někdy mu/jí vyjdou části hodiny, ale často působí napjatě nebo má strach, že něco pokazí.}
{\teacherName ve výuce působí uvolněně a sebevědomě. Baví ho/ji to a má svůj styl.}

\vfill
\rule{\columnwidth}{0.4pt}
\vfill

\rubriccriterion{Formativní zpětná vazba}
{\teacherName nedává osobní zpětnou vazbu studentům.}
{\teacherName se snaží studentům zpětnou vazbu dávat, aby mohli růst. Není ji ale dost, nebo to nedělá efektivně. Studenti tuto zpětnou vazbu někdy nevnímají jako podporu a projev respektu.}
{\teacherName se studenty ve výuce interaguje tak, že dostávají průběžně formativní zpětnou vazbu. Chápou tedy, co jim jde, kde dělají chyby a jak se mohou zlepšovat. Studenti zároveň cítí, že jsou respektováni a zpětné vazby se nebojí.}

\vfill
\rule{\columnwidth}{0.4pt}
\vfill

\rubriccriterion{Jasné zadávání pokynů a úloh}
{\teacherName působí, že nijak zvlášť nepřemýšlí o tom, jak zadávat pokyny nebo úlohy.}
{Stává se, že \teacherName zadá pokyn nebo úlohu a studenti neví, co dělat, jak začít nebo k čemu mají dojít (co má být výsledkem).}
{Když \teacherName zadává pokyn nebo úlohu, studenti mají jasno v tom, co dělat, jak začít nebo k čemu mají dojít.}

\newpage

\rubriccriterion{Variabilita a inovace ve výuce}
{\teacherName působí, že učí tak, jak mu/ji řekli nebo kopíruje výuku, kterou zažil(a). Vypadá to, že nepřemýšlí o jiných variantách.}
{\teacherName vypadá, že si uvědomuje, že existuje mnoho typů aktivit, které lze při výuce použít. Nezná jich ale dostatek, nedokáže je efektivně zadávat, nebo nemá jasno v tom, proč je používat.}
{\teacherName zná mnoho různých aktivit a výuku skládá tak, aby byla dostatečně pestrá. Vybrané aktivity efektivně procvičují probíranou látku. \teacherName zároveň studenty efektivně zapojuje a zvyšuje jejich motivaci se učit.}

\vfill
\rule{\columnwidth}{0.4pt}
\vfill

\rubriccriterion{Širší kontext výuky}
{\teacherName vypadá, že o širším kontextu výuky a kurzu nepřemýšlí.}
{\teacherName působí, že neví, v~jakém kontextu studenti využijí znalosti a dovednosti, které učí. Nevidí propojení s~dalšími kurzy.}
{\teacherName vypadá, že má jasnou představu o tom, k čemu studenty vede (jaké dovednosti rozvíjí, jaké znalosti chce předat). Ví, proč tyto dovednosti rozvíjí a v~jakém kontextu je studenti v~budoucnu použijí. Vidí širší kontext.}

\vfill
\rule{\columnwidth}{0.4pt}
\vfill

\rubriccriterion{Jasné vysvětlování}
{\teacherName působí, že svoje způsoby vysvětlování nijak nereflektuje.}
{Když \teacherName vysvětluje, běžně vypadá, že si není jistý, zda vysvětluje dobře a zda to studentům pomáhá při pochopení.}
{Když \teacherName vysvětluje teorii, demonstruje řešení a efektivně opravuje chyby. Vypadá to, že se dokáže dobře vžít do toho, jak to vidí student a efektivně pomáhá studentům pochopit problém. Nestává se, že by vysvětloval(a) něco, na co se studenti neptali.}

\vfill
\rule{\columnwidth}{0.4pt}
\vfill

\rubriccriterion{Nastavení prostředí, systémy ve výuce}
{\teacherName vypadá, že o atmosféře ve výuce nepřemýšlí.}
{\teacherName působí, že přemýšlí o nastavení pravidel i atmosféry. Nemá ale jasno v efektech těchto pravidel.}
{\teacherName umí ve výuce vytvořit prostředí, které podporuje efektivní učení. U systémů, které používá (např. bodování, bonbóny, zahajovací rituály) působí, že chápe efekt a uzpůsobuje je podle svých potřeb.}

\columnbreak

\rubriccriterion{Flexibilita, přizpůsobování výuky na místě}
{\teacherName svou výuku odvede podle přípravy a nijak ji na místě nepřizpůsobuje potřebám skupiny.}
{\teacherName působí, že si uvědomuje chvíle, kdy by mohlo být zajímavé nebo užitečné dělat něco jiného, než měl(a) v~plánu. Většinou ale nedokáže v~daný moment vhodně zareagovat.}
{\teacherName dokáže svoji výuku průběžně přizpůsobovat tomu, co se právě děje ve skupině a co studenti potřebují. Má k tomu dostatek nástrojů a dokáže je efektivně použít.}

\vfill
\rule{\columnwidth}{0.4pt}
\vfill

\rubriccriterion{Individuální interakce se studenty}
{\teacherName nedělá konzultace nebo jiné indidivuální interakce.}
{Stává se, že \teacherName si často neví rady při individuální interakci se studentem (např. zkoušení u tabule, konzultace). Interakce neprobíhá efektivně nebo se student cítí zastrašen.}
{Při interakcích s jednotlivci (např. zkoušení, konzultace) \teacherName efektivně využívá čas. Studenti konzultují rádi a je to pro ně užitečné. Z učitele cítí respekt a podporu.}

\vfill
\rule{\columnwidth}{0.4pt}
\vfill

\rubriccriterion{Skupinová práce, management skupiny}
{\teacherName nedělí skupinu do menších skupinek.}
{\teacherName vypadá, že si uvědomuje možnosti práce ve dvojicích či menších skupinách. Vypadá to, že tuší, že by toho mohl(a) lépe využívat, hledá jak.}
{\teacherName působí, že má jasno v tom, kdy chce pracovat s~celou skupinou, kdy s~jednotlivci a kdy se skupinkami. Rozdělení do menších skupin efektivně používá, když je to potřeba. Ve vhodných případech zadává interakce mezi skupinami.}

\vfill
\rule{\columnwidth}{0.4pt}
\vfill

\rubriccriterion{Čtení atmosféry ve skupině}
{\teacherName působí, že skupinu ve své výuce nesleduje, pozornost věnuje pouze obsahu.}
{\teacherName vypadá, že si uvědomuje, že mu/ji skupina vysílá signály a že by bylo dobré jim rozumět a využít je pro efektivní vedení výuky. Ve výuce to ale dokáže jen výjimečně.}
{\teacherName dokáže dobře odhadnout naladění skupiny. Vypadá, že má jasno v tom, co se ve skupině studentů děje (např. únava, nadšení, obava).}

\end{multicols*}

\end{document}
